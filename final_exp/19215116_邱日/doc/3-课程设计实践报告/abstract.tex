\begin{center}
	\zihao{3} \heiti	基于x86架构的操作系统之文件系统设计与实现	\end{center}
\begin{center}
	\zihao{-4} \fangsong	计算机科学与技术专业学生 \quad 邱日\\	指导教师 \quad 姜海燕	\end{center}

\zihao{5}
\addcontentsline{toc}{chapter}{摘要}
\noindent{\heiti 摘要:}{\songti 
此次课程设计的目的在于通过在x86架构的计算机上构建简易而真实的操作系统内核,重点实现文件系统部分,来加深
对操作系统文件系统管理原理的认识,乃至对操作系统整体的理解.

在实现上,首先利用grub加载我的内核镜像,然后设置函数栈大小,段的管理上设置好gdt表,初始化idt表,采用连续内存分配.
整个内核具体涉及到处理器管理,中断和异常的处理,系统调用的实现;涉及存储管理,基于连续空间分配和回收存储空间;
涉及设备管理,编写了键盘驱动、字符显示设备的驱动、ATA硬盘驱动,内核能够操纵屏幕键盘和读写硬盘;在文件系统的实现上,
对inode索引节点采用基于位图的空间分配回收,对数据区采用基于成组链接的磁盘空间分配与回收,采用多级目录,三级索引.

通过测试用例的验证,本系统在文件系统方面实现了预期功能,"学中干,干中学",理论实践相结合,自己也提高了操作系统方面
综合能力.
} 

\addcontentsline{toc}{chapter}{关键词} 
\noindent{\heiti 关键词:}{\songti Linux; 文件系统; inode; 超级块;成组链接 } %要具体•避免罕见缩写词和一般性词汇

\vspace{135pt}

% \begin{center}
% 	\zihao{3} Design and implement of a toy-like operating system--"RiOS" \end{center}
% \begin{center}
% 	\zihao{-4} Student majoring in  computer science and technology \quad Ri Qiu\\Tutor\quad Haiyan Jiang\end{center}

% \zihao{5}
% \addcontentsline{toc}{chapter}{Abstract}
% {\renewcommand\baselinestretch{1}\selectfont
% \noindent\textbf{Abstract: }我是摘要
% 	\par}

% \vspace{-0.2\baselineskip}
% \addcontentsline{toc}{chapter}{Key words}
% \noindent\textbf{Key words: } 三; 到; 五; 个; 关键词

% \vspace{15pt}
\clearpage